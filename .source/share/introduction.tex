\documentclass[12pt]{beamer}

\newcommand{\emp}{\noindent\fboxsep=0pt}
\usefonttheme{serif}

\usetheme{Darmstadt}
\usepackage[utf8]{inputenc}
\usepackage[UTF8,zihao=-4]{ctexcap}
\usepackage[T1]{fontenc}
\usepackage{lmodern}
\usepackage{wrapfig}
\usepackage{fontspec}
\usepackage{color}
\usepackage{fancyhdr}
\usepackage{setspace}
\usepackage{caption}
\usepackage{mathptmx}
\usepackage{amsmath}
\usepackage{amssymb}
\usepackage{amstext}
\usepackage{geometry}
\usepackage{graphicx}
\usepackage{pifont}
\usepackage{float}
\usepackage{newclude}
\usepackage{subfig}
\usepackage{paralist}
\usepackage{booktabs} % 允许在表中使用\toprule、\ midrule和\ bottomrule% 定义正文字体
\usepackage[bookmarks=true]{hyperref}
\usepackage{tabularx}
%\usepackage{listings}


\setromanfont{Times New Roman}	% 将西文字体设置为 Times New Roman
% \setCJKfamilyfont{zhkai}{[SIMKAI.TTF]}
% \newcommand*{\kaiti}{\CJKfamily{zhkai}}

% 定义 caption
\DeclareCaptionFont{kaiticaption}{\kaishu \small}	% 定义下面三个caption的font的kaiticaption的具体格式
\captionsetup[figure]{font=small,labelsep=quad,skip=0.5ex,labelfont=bf,font=kaiticaption}	% 设置图片的 caption 格式 % 想要标题换行后居中,可以添加justification=centering
\captionsetup[table]{font=small,labelsep=quad,skip=0.5ex,labelfont=bf,font=kaiticaption}	% 设置表格的 caption 格式
\captionsetup[subfloat]{font=small,labelsep=quad,skip=0.5ex,labelfont=bf,font=kaiticaption}
\captionsetup[equation]{font=small,labelsep=quad,skip=0.5ex,labelfont=bf,font=kaiticaption}


% 设置图片,表格,公式编号格式
\renewcommand{\thetable}{\thesection{}-\arabic{table}}	% \thetable 表示设置的是表格的编号格式, 后面括号的内容为编号格式:章节号-表格的序号
\renewcommand{\thefigure}{\thesection{}-\arabic{figure}}
\renewcommand{\theequation}{\thesection{}-\arabic{equation}}


% 设置代码块
%\definecolor{commentcolor}{RGB}{85,139,78}
%\definecolor{stringcolor}{RGB}{206,145,108}
%\definecolor{keywordcolor}{RGB}{34,34,250}
%\lstset{
%	language=Matlab, % 默认代码语言.
%	basicstyle=\footnotesize,
%	numbers=left, %设置行号位置
%	numberstyle=\tiny, %设置行号大小
%	commentstyle=\color{commentcolor},	%注释颜色
%	keywordstyle=\color{keywordcolor},	%关键词颜色
%	stringstyle=\color{stringcolor},	%字符串颜色
%	frame=single, %设置边框格式
%	escapeinside=``, %逃逸字符(1左面的键),用于显示中文
%	breaklines = True, % 自动折行
%	breakatwhitespace = True, % 自动折行时打断单词
%	extendedchars=false, %解决代码跨页时,章节标题,页眉等汉字不显示的问题
%	xleftmargin=2em,xrightmargin=2em, aboveskip=1em, %设置边距
%	tabsize=4, %设置tab空格数
%	showspaces=false %不显示空格
%}
